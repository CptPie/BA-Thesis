\chapter{RISC-V as a platform}\label{cha:riscV}

One fundamental decision necessary for a Just-in-time compiler regards the platform that shall be supported. 
While that does not massively influence the general design and in principle can be changed fairly easy, the target platform may still impose a few restrictions. 
When talking about target platforms in the context of compilers, we are mainly talking about CPU architectures and more specifically their respective instruction set architectures (ISA). 

The two fairly obvious choices would be the x86 architecture by Intel or the AArch64 ARM architecture due to the widespread usage of their respective chips. 
Both of these would without any obvious problems be capable to handle the instructions specified in Smalltalk-80, as the virtual machine only assumes word sizes of 16 bit, which both of these architectures can support trivially. 

At the beginning of this project, the targeted ISA has been AArch64. The reasoning behind this was due to the angle, from which this whole project started. 
The initial idea was to take the crosstalk project by Michael Engel \cite{crosstalk}, which ported Dan Banays Smalltalk-80 implementation \cite{dbanayST} along with all dependencies to the Raspberry Pi, and implement a JIT compiler on top of the existing implementation.
Since the Raspberry Pi boards, at least at the time, primarily use ARM chips, AArch64 was the obvious choice as a target platform. 

While working on the project an other alternative emerged and gained my interest, the open ISA RISC-V.
Working with RISC-V is very attractive for a few reasons. 
Similar to AArch64 it is, as the name already implies, a reduced instruction set computer (RISC) architecture, which means that there are only a limited amount of instructions available.
In the case of one of the most basic RISC-V ISA (rv32i) there are only 48 instructions defined.
While this can result in the necessity of longer, more complex machine code for a given functionality, this compact instruction set architecture allows an easier understanding on what actually happens.
At the same time I also took an operating systems course using RISC-V.
This, tied with other university activities related to this ISA, reinforced the decision to eventually choose the RISC-V ISA over AArch64. 

Realistically speaking, the choice of the target platform only really matters to this project in the final stage of the JIT compilers development, translating the interpreted instructions into machine code.
At the point of transitioning from AArch64 to RISC-V, this stage was still quite some way away, hence the switch did not have huge influence in how this project eventually turned out.
