\chapter{RISC-V as a platform}\label{cha:riscV}

One fundamental decision necessary for a \jit{} compiler regards the platform that shall be supported. 
While that does not massively influence the general design and can, in principle, be changed fairly easily, the target platform may still impose a few restrictions.
When discussing target platforms in the context of compilers, we are mainly referring to CPU architectures, specifically their respective instruction set architectures (ISA).

Due to the widespread usage of their respective chips, Intel's x86 architecture and the AArch64 ARM architecture are the two fairly obvious choices.
Both of these would be capable of handling the instructions specified in Smalltalk-80 without any obvious problems, as the virtual machine only assumes word sizes of 16 bits, which both of these architectures can support trivially.

At the beginning of this project, the targeted ISA was AArch64. This was because of the idea from which this whole project started.
The initial idea was to take the crosstalk project by Michael Engel \cite{crosstalk}, which ported Dan Banays Smalltalk-80 implementation \cite{dbanayST} along with all dependencies to the Raspberry Pi, and implement a \jit{} compiler on top of the existing implementation.
Since the Raspberry Pi boards, at least at the time, primarily used ARM chips, AArch64 was the obvious choice as a target platform. 

While working on the project, another alternative emerged and gained my interest: the open ISA RISC-V.
Working with RISC-V is very attractive for a few reasons. 
Similar to AArch64, it is, as the name implies, a reduced instruction set computer (RISC) architecture, which means that there are only a limited number of instructions available.
Only 48 instructions are defined for one of the most basic RISC-V ISAs (rv32i).
While this can result in the necessity of bigger, more complex machine code for a given functionality, this compact instruction set architecture allows a more straightforward understanding of what happens within the CPU.
At the same time, I also took an operating systems course using RISC-V.
This, tied with other university activities related to this ISA, reinforced the decision to choose the RISC-V ISA over AArch64.

Realistically speaking, the choice of the target platform only really matters to this project in the final stage of the \jit{} compiler's development, translating the interpreted instructions into machine code.
This stage was still quite some way away when transitioning from AArch64 to RISC-V; hence, the switch did not have a huge influence on how this project eventually turned out.
