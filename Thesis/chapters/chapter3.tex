\chapter{Smalltalk-80}\label{cha:smalltalk}
In this chapter, a description of the Smalltalk-80 ecosystem will be given. 
Unless stated otherwise, the primary source of information is the language specification by Goldberg and Robson, also referred to as the \enquote{Bluebook} because of its blue cover \cite{bluebook1983}.

Fundamentally, Smalltalk-80 is a purely object-orientated programming language, meaning that even primitive types such as numbers are represented by objects. 
As usual, with object-orientation, inheritance is also possible.
While this categorization is correct, narrowing the language down to object-orientation is also misleading. 
A fundamental aspect of the design of Smalltalk-80 revolves around message passing. 
When calling a procedure (a method or function in the terminology of Smalltalk-80), a message is sent to the class of the object the procedure affects. This is called the \enquote{receiver}.
The receiver then looks up the corresponding functionality within its memory context or, if applicable, in the context of its parent class.
After a successful lookup and execution of the procedure, the result will be passed back to the caller, the so-called \enquote{sender}. 

To illustrate this fundamental concept of sender, receiver and message passing, consider calculating the sum of 3 and 4. 
The example is taken directly from Wolczko \cite{Wolczko1984}.

In the syntax of Smalltalk-80, this is written as \stcode{3+4}.
This is to be read from left to right, so on the number object \stcode{3}, the \enquote{selector} \stcode{+} is invoked with the argument of the number object \stcode{4}.
So the current context, the one that actually wants to receive the result, sends a message with the selector \stcode{+} (standing for arithmetic addition) to the underlying class object for the receiver, in this case, the number object \stcode{3}. 
The selector \stcode{+} also takes one argument, which in this case is a reference to the object \stcode{4}. 

To illustrate the following explanation further, see \Cref{fig:illust3+4}. The boxes in this figure represent the boundaries of a context, and the arrows indicate direct or implicit communication between contexts.

When receiving the message (A), the object \stcode{3} looks up the selector in its class definition (B). 
The definition defines a second argument \stcode{aNumber}, the second number for the addition. The implicit call (C) fetches the reference of this argument (D) and forwards it to the context of \stcode{3} (E).
Now, the object \stcode{3} performs the defined procedure (in this case, the arithmetic addition of 3 and 4). 
As a result of this addition, a reference to the number object \stcode{7} is created (F) and returned to the calling context, the sender (G). 

\image[h]{\textwidth}{illust3+4}{Illustration of the context switching and message passing of the procedure \stcode{3+4}}{illust3+4}

The virtual machine operates like a classic stack machine. Each context has an internal stack. When switching contexts, the receiver's stack is initialized with a reference to the sender. Instructions push and pop values to and from the active stack, store values in memory, or are used to trigger the transmission of a message between contexts. 

\section{Structure of Instructions}
In total the Smalltalk-80 virtual machine recognizes 248 bytecode encodings. These bytecodes are encoded within one byte and numbered 0 through 255, with 8 encodings being left unused. 
While this amount of bytecodes seems relatively high, there are only 29 real instructions. 

The vast majority of instructions utilize parts of its bytecode for the argument to the instruction. For example, the instruction to push a constant to the stack is only specified by the three highest bits in the form of \texttt{001}. The remaining 5 bits encode the constant number that should be pushed to the stack. 

Alternatively, there are instructions that are encoded utilizing the full 8 bits. These instructions take one or two complete bytes as input. In these cases, the virtual machine just treats the following bytes as arguments to the instruction. 

Lastly, there are three long jump instructions (160 to 175) that apply both of these encodings.

\Cref{fig:st-bytecodes} shows a table of all Smalltalk-80 instructions, their bytecode ranges and their encoding patterns.The table is taken directly from the language specification.

\image[h]{\textwidth}{st-bytecodes}{Encoding table for the Smalltalk-80 bytecodes}{st-bytecodes}

For the purpose of this thesis, we can simplify the categorization of these instructions into four groups: 
\begin{itemize}
  \item Stack bytecodes: These instructions directly modify the stack. 
  \item Return bytecodes: These instructions are used to return from one context to the sender. They are used to return the result from a message. 
  \item Send bytecodes: The inverse of the return bytecodes, they send a message to a receiver's selector.
  \item Jump bytecodes: This groups together the five explicit jump instructions.
\end{itemize}

Of these, the last three types are especially relevant for the implementation of a \jit{} compiler. 
This is because the return, send and jump instructions all result in a context switch, constituting the end of a \bb{}.

\section{Implementation of the interpreter}
Besides the language specification, the Bluebook also describes the implementation details for a conforming Smalltalk-80 interpreter. 
To understand how the virtual machine works it is useful to first establish the used terminology. 

\paragraph{Context}
A context is a complete copy of the interpreter state at a specific point in time. Instructions may result in changes of the state, hence saving the existing state may be necessary depending on the effects of an instruction. 
A good example of an instruction where the state needs to be saved is the sending of a message. 
When the context is changed, the original context gets saved in memory and a pointer to the state is put on the stack, then the new context gets loaded into the interpreter's state and execution continues. When the new context finishes, if applicable, the results are put on the stack of the original context and the original context is loaded back up.

\paragraph{Location} 
A location is the position within the interpreted code. In other languages this would correspond to line within a given file.
Smalltalk-80 has no concept of files within the virtual machine, hence a more complex system is necessary to keep track of which instruction to interpret next.
To fetch an instruction, the interpreter needs to process a triplet of variables which gets updated by the interpreter when necessary. 
This triplet consists of the pointer to the active context, the pointer to the method that is currently executed and the instruction pointer within the method.

\paragraph{CompiledMethod}
Smalltalk-80 differentiates between the source code of a method and its compiled representation. In this context, \enquote{compiled} does not relate to the aforementioned translation to native machine code, rather to the addition to the currently running interpretation scope. As Smalltalk-80 is constantly self-evaluating its source code, once a method gets interpreted successfully it can be executed. This verified piece of code is then called \emph{CompiledMethod}.

Having found and decoded the instruction as described in above, the Bluebook's description continues to the \enquote{dispatch} stage. The integer representation of the bytecode is passed through a series of methods, and finally arrive at the concrete implementation of the instruction. This function implementation then applies the necessary operations to modify the interpreter state according to the functionality of the instruction.

As with any other interpreter, it is necessary to manage a state within for the interpretation of Smalltalk-80.
The state consists of:
\begin{itemize}
  \item a pointer to the \emph{CompiledMethod}, whose bytecodes are executed, also called the method pointer
  \item the location of the next bytecode within the \emph{CompiledMethod}, the instruction pointer 
  \item the receiver and arguments of the message that invoked the \emph{CompiledMethod}
  \item temporary variables assigned within the method 
  \item the interpreter stack.
\end{itemize}

The description provided in the Bluebook lacks specifics regarding how to draw the graphical user interface or how to manage the virtual machine's memory. This is because the virtual machine is envisioned to be as independent as possible from the machine hosting it.
