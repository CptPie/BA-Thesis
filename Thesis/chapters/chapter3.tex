\chapter{Smalltalk-80}\label{cha:chapter3}
In this chapter a description of the Smalltalk-80 ecosystem will be given. 
Unless stated otherwise, the main source of information is the language specification by Goldberg and Robson, also referred to as the \enquote{Bluebook} because of its blue cover \cite{bluebook1983}.

Fundamentally Smalltalk-80 is an purely object orientated programming language, meaning that even primitive types such as numbers are represented by an object. 
As usual with object orientation, inheritance is also possible.
While this is certainly correct, narrowing the language down to object orientation is also partly misleading. 
A big aspect of the design of Smalltalk-80 revolves around message passing. 
When calling a procedure (a method or function in the terminology of Smalltalk-80), a message is sent to the class of the object the procedure is affecting, this is called the \enquote{receiver}.
The receiver then looks up the corresponding functionality within its memory context or, if applicable, in the context of its parent class.
After a successful lookup and execution of the procedure, the result will be passed back to the caller, the so called \enquote{sender}. 

To illustrate this fundamental concept of sender, receiver and message passing consider the simple task of calculating the sum of 3 and 4. 
The example is taken directly from Wolczko \cite{Wolczko1984}.

In the syntax of Smalltalk-80, this is simply written as \stcode{3+4}.
This is to be read from left to right, so on the number object \stcode{3} the \enquote{selector} \stcode{+} is invoked with the argument of the number object \stcode{4}.
So the current context, the one which actually wants to get the result, sends a message with the selector \stcode{+} (standing for arithmetic addition) to the underlying class object for the receiver, in this case the number object \stcode{3}. 
The selector \stcode{+} also takes one argument, which in this case is a reference to the object \stcode{4}. 

To illustrate the following explanation further, see \Cref{fig:illust3+4}. The boxes in this figure represent the boundaries of a context, the arrows indicate direct or implicit communication between contexts.

When receiving the message (1), the object \stcode{3} looks up the selector in its class definition (2). 
The definition defines a second argument \stcode{aNumber}, the second number for the addition. The implicit call (3) fetches the reference of this argument (4) and forwards it to the context of \stcode{3} (5).
Now the object \stcode{3} performs the defined procedure (in this case the arithmetic addition of 3 and 4). 
As a result of this addition, a reference to the number object \stcode{7} is created (6) and returned to the calling context, the sender (7). 

\image[h]{\textwidth}{illust3+4}{Illustration of the context switching and message passing of the procedure \stcode{3+4}}{illust3+4}

\section{Structure of Instructions}

\section{Implementation of the interpreter}
